\documentclass[conference]{IEEEtran}
\IEEEoverridecommandlockouts
\usepackage{graphicx}
\usepackage{cite}
\usepackage{amsmath,amssymb,amsfonts}
\usepackage{algorithmic}
\usepackage{textcomp}
\usepackage{xcolor}

\def\BibTeX{{\rm B\kern-.05em{\sc i\kern-.025em b}\kern-.08em
    T\kern-.1667em\lower.7ex\hbox{E}\kern-.125emX}}

\begin{document}

\title{Sistema de Prevenção de Falhas em Motores Industriais Usando Visão Computacional, CNN e Machine Learning}

\author{\IEEEauthorblockN{Autor 1\IEEEauthorrefmark{1}, Autor 2\IEEEauthorrefmark{2}, Autor 3\IEEEauthorrefmark{3}}
\IEEEauthorblockA{\IEEEauthorrefmark{1}Afiliação do Autor 1\\
Email: email\_autor1@exemplo.com}
\IEEEauthorblockA{\IEEEauthorrefmark{2}Afiliação do Autor 2\\
Email: email\_autor2@exemplo.com}
\IEEEauthorblockA{\IEEEauthorrefmark{3}Afiliação do Autor 3\\
Email: email\_autor3@exemplo.com}
}

\maketitle

\begin{abstract}
Este artigo apresenta um sistema de prevenção de falhas em motores industriais utilizando visão computacional, Redes Neurais Convolucionais (CNNs) e aprendizado de máquina. O sistema captura imagens do motor em operação e utiliza uma CNN para extrair características relevantes. Essas características são então usadas por um modelo de aprendizado de máquina para classificar o estado do motor e prever possíveis falhas. Os resultados experimentais demonstram a eficácia do sistema proposto na detecção precoce de anomalias, permitindo a manutenção preventiva e reduzindo o tempo de inatividade.
\end{abstract}

\begin{IEEEkeywords}
Prevenção de falhas, motores industriais, visão computacional, CNN, aprendizado de máquina.
\end{IEEEkeywords}

\section{Introdução}
A indústria 4.0 demanda sistemas cada vez mais eficientes e confiáveis. Falhas em motores industriais podem causar paradas na produção, gerando prejuízos significativos.  Este trabalho propõe um sistema para prevenção de falhas que utiliza visão computacional...

\section{Metodologia}
A metodologia proposta consiste em três etapas principais: aquisição de imagens, extração de características e classificação.

\subsection{Aquisição de Imagens}
As imagens são capturadas por câmeras posicionadas...

\subsection{Extração de Características}
Uma CNN pré-treinada, como a ..., é utilizada para extrair características relevantes das imagens. O modelo é ajustado...

\subsection{Classificação}
Um classificador, como uma Máquina de Vetores de Suporte (SVM) ou uma Random Forest, é treinado com as características extraídas pela CNN. O classificador...

\section{Resultados Experimentais}
Os experimentos foram realizados utilizando um conjunto de dados contendo imagens de motores... Os resultados obtidos mostram que o sistema proposto alcança uma acurácia de ... na detecção de falhas.

\section{Conclusão}
Este artigo apresentou um sistema de prevenção de falhas em motores industriais baseado em visão computacional, CNN e aprendizado de máquina. Os resultados demonstram a eficácia do sistema na detecção precoce de anomalias. Trabalhos futuros incluem...

\section*{Agradecimentos}
(Opcional) Agradecimentos a órgãos de fomento, etc.

\bibliographystyle{IEEEtran}
\bibliography{references}

\end{document}